\chapter{Raspberry Pi - GPIO Programming}
\index{Raspberry Pi}
RTB supports the on-board GPIO hardware in the Raspberry Pi computer.

\section{PinMode}
\index{Raspberry Pi!PinMode}
This configures the mode of a pin on the Pi's GPIO. It takes
an argument which specifies the mode of the pin - input, output or PWM
output. 
\begin{verbatim}
  220 PinMode (4, 0)
\end{verbatim}
In this example, we're setting pin 4 to be used for input. The modes are;
\begin{description}
\item[0] Input
\item[1] Output
\item[2] PWM Output
\end{description}

\section{DigitalRead}
\index{Raspberry Pi!DigitalRead}
This function allows you to read the state of a digital pin on the
Raspberry Pi. You may need to set the pin mode beforehand to make
sure it's
configured as an input device. It will return {\tt TRUE} or {\tt FALSE}
to indicate an input being high or low respectively.
\begin{verbatim}
  200 PinMode (12, 0) // Set pin 12 to input
  210 if DigitalRead (12) THEN PROC ButtonPushed
\end{verbatim}

\section{DigitalWrite}
\index{Raspberry Pi!DigitalWrite}
This procedure sets a digital pin to the supplied value - 0 for off or
1 for on. As with {\tt DigitalRead}, you may need to set the pin mode
(to output) beforehand.
\begin{verbatim}
  310 PinMode (2, 1) // Set pin 2 to output mode
  320 DigitalWrite (2, 1) // Set output High (on)
  330 Wait (1)
  330 DigitalWrite (2, 0) // Set output Low (off)
\end{verbatim} 

\section{PwmWrite}
\index{Raspberry Pi!PwmWrite}
This procedure outputs a PWM waveform on the selected pin. The pin must
be configured for PWM mode beforehand.
The value set should be between 0 and 1024.
\begin{verbatim}
  310 PinMode (1, 2) // Set pin 1 to PWM output mode
  320 PwmWrite (11, 200)
\end{verbatim} 
