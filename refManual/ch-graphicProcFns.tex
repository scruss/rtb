\chapter{Graphical Procedures and Functions}\index{Graphics}

RTB supports two types of graphics; traditional Cartesian ($x,y$) type
graphics with functions to plot points, lines, etc. and turtle graphics
which can sometimes be easier to teach/demonstrate to younger people.

RTB normally works in the highest graphical resolution available to it,
but it also supports two effective resolutions - the high resolution
is the native resolution of the display you are using - e.g. on a PC
monitor this may be up to 1280x1024, or 1920x1080 on an HD monitor. It
also supports a lower resolution where each low resolution pixel is
really 8x8 high resolution pixels.

Your programs should use the {\tt GWIDTH} and {\tt GHEIGHT} system
variables to scale their output to the current display size. Assuming
a fixed resolution may not be advisable for programs designed to run on
systems other than your own.

\section{System Variables}
\begin{description}
\item[{\tt COLOUR}]\index{COLOUR} This can be assigned to, or
read from, and represents the current plotting colour.

\item[{\tt GHEIGHT}]\index{GHEIGHT} This can be read to find the current
height of the display in either high resolution or low resolution pixels.

\item[{\tt GWIDTH}]\index{GWIDTH} This can be read to find current
width of the display in either high resolution or low resolution pixels.

\item[{\tt TANGLE}]\index{TANGLE} This can be read or assigned to
and represents the current angle of the turtle when using turtle
graphics mode.
\end{description}

See also the colour names and note the current angle modes when
using turtle graphics or any transcendental functions to plot
graphics.

\section{General Graphics}
\begin{description}
\item[{\tt GR}]\index{GR}
This clears the screen and initialises low-resolution graphics mode.
\item[{\tt HGR}]\index{HGR}
This clears the screen and initialises high-resolution graphics mode.
\item[{\tt SaveScreen (filename\$)}]\index{SaveScreen}
This takes a snapshot of the current screen and saves it to the filename given.
\item[{\tt rgbColour (r, g, b)}]\index{rgbColour}
This sets the current graphical plot colour to an RGB (Red, Green, Blue)
value. The values should be from 0 to 255.
\item[{\tt UPDATE}]\index{UPDATE}
When plotting graphics, they are actually plotted to a separate off-screen
working area, so if you write a program that does lots and lots of graphical
actions, then you may not see the display being updated. The {\tt UPDATE} procedure
forces the working area to be copied to the display. An update is also
performed automatically if your program stops for input, or when you
{\tt PRINT} a new line.
\end{description}

\section{Cartesian Graphics}
The graphical origin at (0,0) is at the bottom-left of the
display. Coordinates may be negative and may also be off-screen when
lines are clipped to the edge of the display. The {\tt ORIGIN} command
may be used to change the graphical origin.

\begin{description}
\item[{\tt PLOT (x, y)}]\index{PLOT}
This plots a single pixel in the selected graphics mode in the selected
colour. Note that (0,0) is bottom left.
\item[{\tt HLINE (x1, x2, y)}]\index{HLINE}
Draws a horizontal line on row {\tt y}, from column {\tt x1} to column {\tt x2}.
\item[{\tt VLINE (y1, y2, x)}]\index{VLINE}
Draws a vertical line on column {\tt x}, from row {\tt y1} to row {\tt y2}.
\item[{\tt LINE (x1, y1, x2, y2)}]\index{LINE}
Draws a line from point {\tt x1, y1} to {\tt x2, y2}.
\item[{\tt LINETO (x1, y1)}]\index{LINETO}
Draws a line from the last point plotted (by the {\tt PLOT} or {\tt LINE}
procedures to point {\tt x1, y1}.

\item[{\tt CIRCLE (cx, cy, r, f)}]\index{CIRCLE}
Draws a circle at ({\tt cx,cy}) with radius {\tt r}. The final parameter,
{\tt f} is either TRUE or FALSE, and specifies filled (TRUE) or outline
(FALSE).

\item[{\tt ELLIPSE (cx, cy, xr, yr, r, f)}]\index{ELLIPSE}
Draws an ellipse at ({\tt cx,cy}) with $x$ radius {\tt xr} and $y$
radius {\tt yr}. The final parameter, {\tt f} is either TRUE or FALSE,
and specifies filled (TRUE) or outline (FALSE).

\item[{\tt TRIANGLE (x1, y1, x2, y2, x3, y3, f)}]\index{TRIANGLE}
Draws a triangle with its corners at the three given points.
The final parameter, {\tt f} is either TRUE or FALSE, and specifies
filled (TRUE) or outline (FALSE).

\item[{\tt PolyStart}]\index{PolyStart} This marks the start of
drawing a filled polygon.

\item[{\tt PolyPlot (x, y)}]\index{PolyPlot}
This remembers the given X,Y coordinates as part of a filled polygon.
Nothing is actually drawn on the screen until the {\tt PolyEnd}
instruction is executed. Polygons can have a maximum of 64 points.

\item[{\tt PolyEnd}]\index{PolyStart} This marks the end of
drawing a polygon. When this is called, the stored points will
be plotted on the screen and the polygon will be filled.


\item[{\tt RECT (x, y, w, h, f)}]\index{RECT}
Draws a rectangle at ({\tt x,y}) with width {\tt w} and height {\tt
h}. The final parameter, {\tt f} is either TRUE or FALSE, and specifies
filled (TRUE) or outline (FALSE).

\item[{\tt ORIGIN (x, y)}]\index{ORIGIN}
This changes the graphics origin for the Cartesian plotting
procedures. The {\tt x, y} coordinates are always absolute coordinates
with (0,0) being bottom left.
\end{description}

\section{Turtle Graphics}
When the graphics are initialised with either {\tt GR} or {\tt HGR},
the turtle position is set to the middle of the screen, pointing at
an angle of 0 degrees (up) with the pen lifted. However note that the
turtles position is affected by the {\tt ORIGIN} command above, so you
may have to take this into consideration if using {\tt ORIGIN}.

With the noted effects of the {\tt ORIGIN} procedure,
you may mix turtle and Cartesian graphics without any issues.

\begin{description}
\item[{\tt PENDOWN}]\index{PENDOWN}
This lowers the ``pen'' that the turtle is using to draw with. Nothing will
be drawn until you execute this procedure.
\item[{\tt PENUP}]\index{PENUP}
This lifts the ``pen'' that the turtle uses to draw. You can move the turtle
without drawing while the pan is up.
\item[{\tt MOVE (distance)}]\index{MOVE}
This causes the turtle to move forwards {\tt distance} screen pixels.
\item[{\tt MOVETO (x, y)}]\index{MOVETO}
This moves the turtle to the absolute location ({\tt x, y}). A line will
be drawn if the pen is down.
\item[{\tt LEFT (angle)}]\index{LEFT}
Turns the turtle to the left (counter clockwise) by the given angle.
\item[{\tt RIGHT (angle)}]\index{RIGHT}
Turns the turtle to the right (clockwise) by the given angle.
\end{description}

\section{Sprite Procedures and Functions}
Sprites are (usually) small rectangular or square bitmap images which
you can move round the screen under program control. You can create them
using one of the many graphical image creation packages available. The
file-format is 24 bits per pixel BMP, and a colour of 254,254,254 is
transparent.

\begin{description}
\item[{\tt LoadSprite (filename\$)}]\index{LoadSprite}
This loads a sprite from the supplied fine into memory and returns a
handle to the internal sprite data. You need to use the number returned
in all future sprite handling functions/procedures.
\item[{\tt PlotSprite (sprite, x, y)}]\index{PlotSprite}
This plots the given sprite {\tt sprite} at the suppled {\tt x, y}
coordinates. The coordinates specify the bottom-left corner of the
bounding rectangle of the sprite.
\item[{\tt DelSprite}]\index{DelSprite}
This removes a sprite from the screen.
\end{description}

You do not have to erase a sprite from the screen when you move it,
just call {\tt PlotSprite} with the new coordinates.
