\chapter{Serial Port Programming}
\index{Serial Ports}
RTB supports a few instructions to allow you to send and receive data
over serial ports. These go by various names, a common one being RS232.

Moderns PCs often lack a directly connected serial port, so you may have
to use a USB connected device. Using them is the same whether directly
connected or via USB. The important thing to know is the name of the
serial device. For directly connected devices, this might be something
like "/dev/ttyS1", or for USB devices it might be something like
"/dev/ttyUSB0". Occasionally you have find a "/dev/ttyACM0" for some
older USB devices.

\section{SOPEN}
\index{Serial Ports!SOPEN}
This opens a serial device and makes it available for our use. It
takes the name of the serial port and the speed as an argument and
returns a number (the {\em handle}) of the device. We can use this
handle to reference the device and allow us to open several devices at
once. Example:
\begin{verbatim}
  100 arduino = SOPEN ("/dev/ttyUSB0", 115200)
\end{verbatim}
The following baud rates are recognised: {\tt 50, 75, 110, 134, 150,200,
300, 600, 1200, 1800, 2400, 19200, 38400 57600, 115200 and 230400},
but do check your local PC and devices capabilities. The device is
always opened with the data format set to 8 data bits, 1 stop bit and
no parity. All handshaking is turned off.

\section{SCLOSE}
\index{Serial Ports!SCLOSE}
This closes a serial port and frees up and resources used by it. It's
not strictly necessary to do this when you end your program, but it is
considered good practice.
\begin{verbatim}
  120 SCLOSE (arduino)
\end{verbatim}

\section{SGET, SGET\$}
\index{Serial Ports!SGET}\index{Serial Ports!SGET\$}
Fetch a single byte of data from an open serial port and return the
data as either a number or a single-character string. This function will
pause program execution for up to 5 seconds if no data is available. If
there is still not data after 5 seconds, the function will return -1 or
an empty string.
\begin{verbatim}
  130 d = SGET (arduino)
  140 key$ = SGET$ (arduino)
\end{verbatim}

\section{SPUT, SPUT\$}
\index{Serial Ports!SPUT}\index{Serial Ports!SPUT\$}
Send a single byte of data, or a string of characters to an open serial port.
\begin{verbatim}
  130 SPUT (arduino, byte)
  140 SPUT$ (arduino, "Hello")
\end{verbatim}

\section{SREADY}
\index{Serial Ports!SREADY}
Returns the number of characters avalable to be read from an open serial
port. This can be used to poll the device to avoid stalling your program
when there is no data avlable to be read.
\begin{verbatim}
  210 IF SREADY (arduino) THEN PROC getSerialData
\end{verbatim}
