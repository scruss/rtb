\chapter{Functions}
\index{Functions}
A function is very much like a predefined mini-program that processes
some data and returns a result. The result can then be used in more
calculations or printed out.

Most functions deal with complex mathematical operations -- if this is
a bit beyond your abilities then just skip over it -- it's not that
important for now, but there are also functions that deal with string
variables.

Functions often take parameters (often called {\em arguments}), and
these must be enclosed them in brackets. 

As well as the functions that are built into RTB, you can also write your
own functions and we'll be looking at how to do that in a later chapter.

\section{Numerical Functions}
\index{Functions!Numerical}
Our first example is the square root function and it's defined as {\tt
SQRT}\index{SQRT}. So to print the square root of nine, ($\sqrt{9}$)
then in immediate mode, enter:

\begin{verbatim}
  ? SQRT (9)
\end{verbatim}
It should print {\tt 3}.  Try this:
\begin{verbatim}
  ? SQRT (9) * SQRT (9)
\end{verbatim}
It should print {\tt 9}. 

Other built-in numerical functions are {\tt SIN}\index{SIN}, {\tt
COS}\index{COS}, {\tt TAN}\index{TAN} and their counterparts,
{\tt ASIN}\index{ASIN}, {\tt ACOS}\index{ACOS} and {\tt
ATAN}\index{ATAN}. These perform the usual trigonometric functions, with
{\tt LOG}\index{LOG} and {\tt EXP}\index{EXP} being natural logarithm
and it's inverse. These are the main ones -- there are a few more,
just look in the reference section.

\subsection{Angles}
\index{Angles}
RTB starts by default using degrees as it's measurement of angles. You can
switch modes using the following commands: {\tt DEG}\index{Angles!DEG},
{\tt RAD}\index{Angles!RAD} or {\tt CLOCK}\index{Angles!CLOCK}. The
first two selects degrees\index{Degrees} or radians\index{Radians} as
the angular unit of measurement (360$^{\circ}$ in a circle, or 2$\pi$
radians) the last selects clock\index{Clock} mode where there are 60
minutes in a full circle. This is intended to make it easier to visualise
angles when using the turtle drawing functions\footnote{Although in these
modern times of digital watches\index{Digital Watch}, who knows what
makes sense anymore!}.

There are 90$^{\circ}$ or $\frac{1}{2}\pi$ radians or 15
minutes in a right angle as this program demonstrates:
\begin{verbatim}
  10 DEG
  20 PRINT SIN (90) // Should print 1
  30 RAD
  40 PRINT SIN (3.1416 / 2)
  50 CLOCK
  60 PRINT SIN (15)
  70 END
\end{verbatim}

If \meek you switch modes when in interactive mode, then it will switch
back to degrees when you run the program, so if you need another
unit of measurement, then put the instruction in the first few lines
of your program.

\section{Random numbers}
\index{Random numbers}
Sometimes you need a random number and RTB provides a function which you
can use in 3 different ways to generate random numbers. The function
is called {\tt RND}\index{Random numbers!RND} and when you give it
an argument of zero (0), then it will return the last random number
generated. If you call it with one (1), then it will generate a random
number in the range {\tt $0 <= RND < 1$}, and if you call it with a
number greater than one then it will generate an integer random number
in the range {\tt $0 <= RND < N$}

The following program will demonstrate:
\begin{verbatim}
   10 FOR i = 1 TO 5 CYCLE
   20   PRINT RND (10)
   30   REPEAT
   40 END
\end{verbatim}
Change the number {\tt 10} above from 0 to 1 then to any other number
to see the effect it has.

The random number starting value (or seed) is re-generated every time you
start RTB from scratch, but you can force the random number generator to
generate the same sequence every time by assigning a number to the {\tt
SEED}\index{Random numbers!SEED} built-in variable (introduced on page
\arabic{seedVariable}) This may seen counter intuitive, but sometimes
knowing the numbers in advance can help when testing a program. If you
assign the value of zero to {\tt SEED} then it will use an internally
generated random number.

\section{String Functions}
\index{Functions!String}
We can concatenate strings together using the plus sign, but in addition
to this, there are a few functions we can use with strings that might
be handy.

{\tt LEN}\index{LEN} -- This returns the length of a string, so:
\begin{verbatim}
  ? LEN ("Hello")
\end{verbatim}
should print {\tt 5}.

Note \meek that functions which return a string have a dollar sign
as part of their name - this is to keep things consistent with string
variables. If it has a dollar then it needs to be assigned to a string,
if not, then it needs to be assigned to a number.

{\tt LEFT\$}\index{LEFT\$}, {\tt MID\$}\index{MID\$}, {\tt
RIGHT\$}\index{RIGHT\$} -- These return parts of a string. For example:
\begin{verbatim}
  10 test$ = "Hello, world"
  20 PRINT LEFT$ (test$, 2);  RIGHT$ (test$, 2);
  30 PRINT MID$ (test$, 5, 2);  RIGHT$ (test$, 5)
  40 END
\end{verbatim}
There are functions to convert a string into a number and back again:
{\tt VAL}\index{VAL} takes a string argument and returns the ``value'' of it as
a number, and {\tt STR\$}\index{STR\$} takes a number and returns a string.
\begin{verbatim}
  10 REM VAL and STR$ check
  20 test = val (str$ (10))
  30 PRINT "Test should be 10. It is: "; test
  40 test$ = str$ (val ("123"))
  50 PRINT "Test should be 123. It is: "; test$
  60 END
\end{verbatim}
These are the main built-in functions. There are a few other functions
that are not listed here. Read the reference section for a complete list.
