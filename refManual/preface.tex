\addcontentsline{toc}{chapter}{Preface}
\section*{Preface}
\index{Preface}

\vspace{3ex}
\noindent
Return to Basic (RTB) is a project to re-create a modern version of
the BASIC interpreter and programming environment popular in the
late 1970s through the 1980s on the popular 8-bit microprocessor
systems such as the Apple ][ series, BBC Micro and Commodore PET
as well as the various Sinclair systems and a whole host of others
too numerous to mention. The intention is to present an easy to use
environment for young (and old) people to learn to write computer
programs using an interactive system that's quick and easy to use.
\index{Apple}\index{BBC}\index{Sinclair}\index{Commodore PET}

BASIC (Beginners All-purpose Symbolic Instruction Code) is regarded as
somewhat old-fashioned in todays world (c2012), but I believe it still
has a place in the teaching and learning of computer programming. It
arguably has many faults and has been criticised for encouraging bad
programming practices, however RTB is a new implementation, capable
of using modern structured programing techniques including named
functions and procedures (which support recursion, if desired) and
structured looping constructs. Also, while this manual demonstrates the
use of the GOTO instruction it does discourage it and quickly moves on
to demonstrating newer techniques. You can even write RTB programs without
line numbers and merge library files of procedures and functions together.

RTB is not intended to be used as a serious programming system --
While it has the capabilities to write a large financial package in, I
do not expect people to write the many, varied and large packages that
were commonly written in BASIC in those early days, but rather to be
used as a way to introduce programming in an easy to understand manner.

RTB features several graphical systems that were popular in those early
microprocessor years -- both a low and high resolution colour graphics
system as well as ``turtle'' graphics with simplified colour and angle
handling. It is more than capable of being used to write simple games
and animations in.

I would like to think that pre-teen children can be introduced to
programming using RTB -- the only prerequisite is a desire to learn and
the ability to type some simple instructions into a computer. It doesn't
even need a mouse!

This manual is intended to be used as a reference manual for RTB. It
lists all the major programming constructs with a small number of examples.
\vfill
{\hfill\em Gordon Henderson, February 2012}
\newpage
\addcontentsline{toc}{chapter}{Trademarks and Acknowledgements}
\index{Trademarks}\index{Acknowledgements}
\section*{Trademarks and Acknowledgements}
Raspberry Pi is a registered trademark of the Raspberry Pi Foundation


\addcontentsline{toc}{chapter}{Warranty and Copying}
\index{Warranty}\index{Copying}
\section*{Warranty and Copying}
\begin{verbatim}
 * This document is part of RTB:
 *	Return To Basic
 *	http://projects.drogon.net/return-to-basic
 *
 *    RTB is free software: you can redistribute it and/or modify
 *    it under the terms of the GNU General Public License as published by
 *    the Free Software Foundation, either version 3 of the License, or
 *    (at your option) any later version.
 *
 *    RTB is distributed in the hope that it will be useful,
 *    but WITHOUT ANY WARRANTY; without even the implied warranty of
 *    MERCHANTABILITY or FITNESS FOR A PARTICULAR PURPOSE.  See the
 *    GNU General Public License for more details.
 *
 *    You should have received a copy of the GNU General Public License
 *    along with RTB.  If not, see <http://www.gnu.org/licenses/>.
\end{verbatim}

