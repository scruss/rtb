\chapter{Numbering Names and Formats}
When numbers get very big or very small, it starts to become hard to
represent them, so we have developed various way to express them. Then
computers came along and things are now somewhat different and potentially
confusing.

The most common way to represent big or small numbers is to write them
in groups of three digits and the International System of Units (SI,
or Syst\`{e}me international d'unit\'{e}s) has a set of standard names
to represent these, however there is a complication. The SI units use
powers of 10 (so 1,000, 1,000,000 and so on), but computers count in
twos (binary), so we've adopted the same names as the decimal units
for computer sizes. A further complication is that some components
manufacturers use the SI units to represent capacities because people
are used to thinking in computers terms, but then find that the decimal
terms are actually smaller. Disk drive manufacturers are frequently
guilty  of this.

The following table gives the most popular definitions you'll encounter:

%\noindent
\begin{tabular}[t]{|c|c|c|c|c|}
\hline
Name	& Kilo		& Mega		& Giga		& Tera			\\
\hline
Decimal	& 1,000		& 1,000,000	& 1,000,000,000	& 1,000,000,000,000	\\
\hline
	& 10$^3$	& 10$^6$	& 10$^9$	& 10$^{12}$		\\
\hline
\hline
Computer& 1024		& 1,048,576	& 1,073,741,824 & 1,099,511,627,776	\\
\hline
	& 2$^{10}$	& 2$^{20}$	& 2$^{30}$	& 2$^{40}$		\\
\hline
\end{tabular}

\noindent
You may see reference to the term {\sl MiB} in some texts - this stands
for Mebi - as in Mebibyte and it used to represent the binary forms. So
2MB is 2,000,000 bytes, but 2MiB is 2,097,152 bytes.

For the most part you can treat the decimal and binary forms as more or
less the same. The MiB format isn't that common yet, but seems to be gaining
popularity in some circles.
