\chapter{File Handling functions}
\index{File Handling}

RTB supports reading and writing of files stored on local non-volatile
media (typically disk drives). There are two main methods of file
accessing; sequential and random-access. When you open a file, it is
assigned a numerical {\em ``handle''} and you need to use this handle when
performing other operations on the file. RTB allows you to open multiple
files at the same time, up to a limit set by the implementation. (Usually
8). Each open file will have a unique handle, so you must keep track of
them when manipulating files.

All data written to, or read from a file is textual in nature. You can not
write binary data in RTB. Numbers are stored as literal strings as if they
have been printed on the screen. (And in particular, the format set by
the {\tt NUMFORMAT} instruction is used when printing numbers into files).

\section{Sequential Access}
\index{File Handling!Sequential}
Sequential file access is straightforward. You open a file, then read from
it, or write to it. You can reset the file pointer to the start of the
file using the {\tt REWIND} instruction, so you can write data to a file,
then {\tt REWIND}, then read the data back again. There is no structure
to the file other than what you write into it yourself. ie. if you want
to read the 15th line in a file, then you need to read the preceding 14
lines unless you know the exact line length of each line, then you can use
the {\tt SEEK} instruction to jump to that line, but if the lines are all
different length, then you need to read the while file from start to end,

\section{Random Access}
\index{File Handling!Random}
Random file access is similar to sequential, however it additionally
allows you to split your file into fixed-size records, then
you can {\tt SEEK} to any of these records at random.

There are no special instructions that make a file any different when in
sequential mode to random mode other than those you code yourself. You
must plan the record size (in bytes or characters) before hand and you
must always seek to the correct record in your file before reading or
writing. Each record is then treated like a small sequential file and you
can read and write into it - however you must make sure that you don't
write more data into the record than you have planned for, otherwise
that data will overflow into the next record with unpredictable results.

Random access files can be wasteful of disk space, but they can also
be a fast an versatile method of storing and retrieving data. Most modern
databases use this technique to store data for fast retrieval.

\section{RTB Instructions and functions}

\subsection*{OPEN (filename\$)}
\index{File Handling!OPEN}
The {\tt OPEN} function opens a file and makes it available for reading
or writing and returns the numeric handle assiciated with the file. The
file is created if it doesn't exist, or if it does exist the file pointer
is positioned at the start of the file. To append data to the end of the
file, you need to use the {\tt FFWD} instruction after opening the file.

\subsection*{CLOSE (handle)}
\index{File Handling!CLOSE}
The {\tt CLOSE} instruction closes a file after use and makes sure all
data is securely written to the storage medium.

\subsection*{EOF (handle)}
\index{File Handling!EOF}
The {\tt EOF} function returns a {\tt TRUE} or {\tt FALSE} indication
of the state of the file pointer when reading the file. It is an error
to try to read past the end of the file, so if you are reading a file
with unknown data in it, then you must check at well defined intervals
(e.g. before each {\tt INPUT\#}.

\subsection*{REWIND (handle)\\FFWD (handle)}
\index{File Handling!REWIND}\index{File Handling!FFWD}
Theese instructions move the file pointer back to the start of the file
or to the end of the file respectively. If you want to append data to
the end of an existing file, then you need to call {\tt FFWD} before
writing the data.

\subsection*{SEEK (handle, offset)}
\index{File Handling!SEEK}
The {\tt SEEK} instruction moves the file pointer to any place in the
file. It can even move the file pointer beyond the end of the file in
which case the file is extended. The argument suppled to {\tt SEEK}
is an absolute number of bytes from the start of the file. If you are
using random access files and want to access the 7th record in the file,
then you need to multiple your record size by 7 to get the final location
to seek to.

\subsection*{PRINT\# handle, \dots}
\index{File Handling!PRINT\#}
The {\tt PRINT\#} instruction acts just like the regular {\tt PRINT}
instruction except that it sends data to the file identified by the
supplied file-handle rather than to the screen. Numbers are formatted
according to the settings of {\tt NUMFORMAT}. It is strongly recommended
to only print one item per line if you are going to read those items
back into an RTB program again.
\begin{verbatim}
  100 myFile = OPEN ("testfile.dat")
  110 FFWD (myFile)
  120 PRINT# myFile, "Appended line"
  130 CLOSE (myFile)
\end{verbatim}

\subsection*{INPUT\# handle, variable}
\index{File Handling!INPUT\#}
This works similarly to the regular {\tt INPUT} instruction, but reads
from the file identified by the supplied file-handle rather than from the
keyboard. Note that unlike the regular keyboard {\tt INPUT} instruction,
{\tt INPUT\#} can only read {\em one} variable at a time.
\begin{verbatim}
  100 myFile = OPEN ("testfile.dat")
  110 WHILE NOT EOF (myFile)
  120   INPUT# myFile, line$
  130    PRINT line$
  140 CLOSE (myFile)
\end{verbatim}
