\chapter{Colours}\index{Colours}

To keep thing simple, RTB has 16 colours pre-definied for your use. You
can use numbers to represent them, or use the built-in names. The
colours are:

\index{Black}\index{Navy}\index{Green}\index{Teal}\index{Maroon}
\index{Purple}\index{Olive}\index{Silver}\index{Grey}\index{Blue}
\index{Lime}\index{Aqua}\index{Red}\index{Pink}\index{Yellow}\index{White}
\begin{center}
\begin{tabular}{|c|c|c|c|c|c|c|c|c|}
\hline
0 & 1 & 2 & 3 & 4 & 5 & 6 & 7\\
\colorbox{rtb-black}{\hspace{10mm}}&
\colorbox{rtb-navy}{\hspace{10mm}}&
\colorbox{rtb-green}{\hspace{10mm}}&
\colorbox{rtb-teal}{\hspace{10mm}}&
\colorbox{rtb-maroon}{\hspace{10mm}}&
\colorbox{rtb-purple}{\hspace{10mm}}&
\colorbox{rtb-olive}{\hspace{10mm}}&
\colorbox{rtb-silver}{\hspace{10mm}}\\
Black & Navy & Green & Teal & Maroon & Purple & Olive & Silver\\
\hline
\hline
8 & 9 & 10 & 11 & 12 & 13 & 14 & 15\\
\colorbox{rtb-grey}{\hspace{10mm}}&
\colorbox{rtb-blue}{\hspace{10mm}}&
\colorbox{rtb-lime}{\hspace{10mm}}&
\colorbox{rtb-aqua}{\hspace{10mm}}&
\colorbox{rtb-red}{\hspace{10mm}}&
\colorbox{rtb-fuchsia}{\hspace{10mm}}&
\colorbox{rtb-yellow}{\hspace{10mm}}&
\colorbox{rtb-white}{\hspace{10mm}}\\
Grey & Blue & Lime & Aqua & Red & Pink & Yellow & White\\
\hline
\end{tabular}
\end{center}
Note that the colours here are just representative and might not be what
you see on the video screen you are using to run RTB on.

These colours are avalable in both text and graphics modes.

If these 16 colours are not sufficient, then it's possible to use
the full colour pallet of the computer system you are using - this is
usually a 24-bit system with 8 bits for each of Red, Green and Blue,
giving a maximum of 16777216 different colous avalable.

To use the full-colour mode, you need to use the built-in procedure
\begin{itemize}
\item {\tt rgbColour (red, green blue)}
\end{itemize}
where the values of {\tt red, green} and {\tt blue} are numbers from 0
to 255 inclusive.

The extended RGB colours are avalable in graphics modes only. Text is
limited to the standard 16 colours.

\section{A Text Colour Demo}
\index{Text Screen!colour Demo}
There should be a program supplied with your RTB installation called
"colours" which will demo the basic colours in text mode. {\tt LOAD
colours} should obtain it for you, but if it's not avalable then:
\begin{samepage}
\begin{verbatim}
  100 REM Colour test program
  110 //
  120 DIM c$(15)
  130 FOR i = 0 TO 15 CYCLE 
  140   READ c$(i)
  150 REPEAT 
  160 //
  170 CLS 
  180 VTAB = 4
  190 PRINT "Colour test program"
  200 PRINT "==================="
  210 VTAB = 10
  220 FOR bg = 0 TO 15 CYCLE 
  230   FOR i = 0 TO 1 CYCLE 
  240     BCOLOUR = 0
  250     TCOLOUR = bg
  260     PRINT c$(bg);  
  270     FOR fg = 0 TO 7 CYCLE 
  280       PROC testcolour(fg + i * 8, bg)
  290     REPEAT 
  300     PRINT 
  310   REPEAT 
  320 REPEAT 
  330 TCOLOUR = 15
  340 BCOLOUR = 0
  345 END 
  350 //
  360 // Procedure to print some text in the colours supplied
  370 //
  380 DEF PROC testcolour(f, b)
  390 TCOLOUR = f
  400 BCOLOUR = b
  410 PRINT c$(f);  
  420 ENDPROC 
  900 //
  901 DATA " Black  ", " Navy   ", " Green  ", " Teal   "
  902 DATA " Maroon ", " Purple ", " Olive  ", " Silver "
  903 DATA " Grey   ", " Blue   ", " Lime   ", " Aqua   "
  904 DATA " Red    ", " Pink   ", " Yellow ", " White  "
\end{verbatim}
\end{samepage}
