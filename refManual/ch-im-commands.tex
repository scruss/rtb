\chapter{Immediate mode commands}\index{Immediate mode commands}

In immediate mode, as well as executing simple print instructions as
well as other instructions you can use within a program, there are several
commands which can only be executed in immediate mode.

\begin{description}
\item[{\tt LIST [[first] last]}:]\index{LIST}
This lists the program stored in memory to the screen. You can pause the
listing with the space-bar and terminate it with the escape key. If {\tt
first} is given, then only that line will be listed. If both {\tt first}
and {\tt last} are given, then lines in that range will be listed.
\item[{\tt SAVE [filename]}:]\index{SAVE}
Saves your program to the local non-volatile storage. {\tt filename} is
the name of the file you wish to save and may not contain spaces. If you
have already saved a file, then you can subsequently execute {\tt SAVE}
without the filename and it will overwrite the last file saved. (This is
``safe'' as it's reset when you load a new program or use the {\tt NEW}
command)
\item[{\tt SAVENN [filename]}:]\index{SAVENN}
Same as the normal {\tt SAVE} command, but saves with No line
Numbers. You can only save without line numbers if your program has no
{\tt GOTO} or {\tt GOSUB} statements and no {\tt RESTORE} statements
with a line-number.
\item[{\tt LOAD filename}:]\index{LOAD}
Loads in a program from the local non-volatile storage. As with {\tt
SAVE}, you need to supply the filename without any quotes.
\item[{\tt DIR [directory]}:]\index{DIR}
Lists the RTB files in your current working directory, or the given
directory (without quotes)
\item[{\tt NEW}:]\index{NEW}
Deletes the program in memory. There is no verification and once it's
gone, it's gone. Remember to save first!
\item[{\tt RUN}:]\index{RUN}
Runs the program in memory. You may give a line number and the
program will start from that number rather than the first line in the
program. Note that using {\tt RUN} will clear all variables.
\item[{\tt CONT}:]\index{CONT}
Continues program execution after a {\tt STOP}\index{STOP}
instruction. Variables are not cleared.
\item[{\tt CLEAR}:]\index{CLEAR}
Clears all variables and deletes all arrays. It also removes any active
sprites from the screen.  Stopped programs may not be continued after
a {\tt CLEAR} command.
\item[{\tt TRON}:]\index{TRON}
Turns line-number tracing on. As each line is executed, it's number is
printed to the text console that RTB was started from.
\item[{\tt TROFF}:]\index{TROFF}
Turns line number tracing off.
\item[{\tt ED linenumber}:]\index{ED}
Edit the line given.
\item[{\tt RENUMBER [start [inc [first [last]]]]}:]\index{RENUMBER}
This renumbers your program - by default it will start at 100 and go
in increments of 10, however you can change this as follows: The {\tt
start} and {\tt inc} parameters specify the new starting line number and
increment, the {\tt first} and {\tt last} parameters specify the first
and last {\em exiting} line numbers to renumber. Using this latter way,
you can move lines in the program, but beware of overlaps.

If an overlap does occur, then renumbering will stop at that point and
you may find your program to be somewhat scrambled\dots Please make sure
you save your program before renumbering!

\item[{\tt VERSION}:]\index{VERSION}
Print the current version of RTB.
\item[{\tt EXIT}:]\index{EXIT}
Exit RTB and return to the environment you started RTB in.
\end{description}

\section{File names}
\index{File names}
The system you run RTB on may have its own rules
about what a filename can look like, and whether the name is
case-sensitive. (ie. UPPER/lower case letters -- are they considered the
same, or different?) In the Linux environment then case is significant. If
you're not sure, just stick to simple names without spaces. If you're
subsequently looking at the files outside the RTB environment note that
the filenames will have the characters {\tt .rtb} appended to it.
