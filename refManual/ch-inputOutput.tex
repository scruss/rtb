\chapter{Input and Output}
\index{Input}\index{Output}

\section{Input}
You can read data into your program with the {\tt INPUT} statement.
It works as follows:

\begin{verbatim}
  100 INPUT test
  110 PRINT test
  199 END
\end{verbatim}
You can only input one item at a time.

When RTB encounters the {\tt INPUT} statement, program execution
stops, a question-mark ({\tt ?}) is printed and it waits for you
to type something. 

It then assigns what you typed to the variable.

If you typed in a string when it was expecting a number, then it
will assign zero to the number.

To stop it printing the question mark, you can optionally give it
a string to print:
\begin{verbatim}
  100 INPUT "Give me a number: ", test
  110 PRINT test
  199 END
\end{verbatim}

{\tt INPUT} reads a line of text in until you press the {\tt ENTER}
key. To just read in a single character, you can use the {\tt GET}
instruction to read the key as a number, or the {\tt GET\$} instruction
to read the key as a string.
\begin{verbatim}
  100 PRINT "Press any key to continue ";
  110 key$ = GET$
  199 END
\end{verbatim}

Finally, it's sometimes handy to be able to see if a key has been pressed
without stopping the program running - e.g. in games and simulations. To
do this, we can use the {\tt INKEY} command. This will return the ASCII
value of the key presses, or -1 if no key has been pressed.
\begin{verbatim}
  100 REM Simple reaction timer
  110 WAIT (2)
  120 WHILE INKEY <> -1 CYCLE 
  130   // Do nothing, just make sure no keys have been pushed
  140 REPEAT 
  150 start = TIME
  160 PRINT "Go!"
  170 WHILE INKEY = -1 CYCLE 
  180   // Do nothing
  190 REPEAT 
  200 etime = TIME
  210 PRINT "Your reaction time is ";  etime - start;  " milliseconds"
  220 END 
\end{verbatim}

\section{Output}
Outputting text to the screen is done via the {\tt PRINT}
command. The {\tt PRINT} command is quite versatile and
will print any combination of numbers and strings separated
by the semicolon ({\tt ;})
\begin{verbatim}
  100 INPUT "Give me a number: ", test
  110 PRINT "The number you gave me was "; test
  199 END
\end{verbatim}
Normally the {\tt PRINT} command will move to the next line of
output at the end of the program statement, but you can suppress
this with a trailing semicolon.
\begin{verbatim}
  100 PRINT "This is on a line on its own"
  110 PRINT "This is at the start ";
  120 PRINT "and this is at the end of the same line"
  199 END
\end{verbatim}
The {\tt PRINT} command may be abbreviated with a question mark ({\tt ?})
\begin{verbatim}
  100 ? "Hello, world"
\end{verbatim}
but if you type {\tt LIST} it will be converted back into the full {\tt
PRINT} instruction however the {\tt ?} way is handy for immediate mode
calculations.

You can affect the way numbers are printed using the {\tt NUMFORMAT}
procedure. This takes 2 arguments, the first specifying the total
number of characters to print and the 2nd the number of characters
after the decimal point.

\begin{verbatim}
  100 NUMFORMAT (6,4)
  110 PRINT PI
  199 END
\end{verbatim}
will print: {\tt 3.146}

Numbers printed this way are right-justified with leading spaces inserted
if required.

{\tt NUMFORMAT (0,0)} restores the output to the general purpose format
used by default.
