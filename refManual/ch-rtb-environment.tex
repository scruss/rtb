\chapter{The RTB environment}

RTB is primarily designed to run on computers running Linux. In the
fullness of time, both MS windows and Apple Mac versions may be made
available, but for now it's Linux only.

On the Raspberry Pi, there are some additional features to make use of
the Pi's GPIO facility.
\index{PC}\index{Apple}\index{Mac}\index{Raspberry Pi}\index{Linux}

Starting RTB may vary from one Linux system another, however opening up
a text (or command/shell) terminal and typing

{\tt rtb}

at the prompt will usually get things going for you.

Whatever the system, once RTB has started it should look the same. There
will be a keyboard to type commands into, this is connected to the main
computer which should be connected to a screen or display of some sort. A
Raspberry Pi may even be connected to your home TV set.

When it's setup and ready, the screen should be clear apart from some
introductory text and a prompt. It will probably look something like this:
\begin{verbatim}
  Return to Basic version 1.0 
  Ready
  >
\end{verbatim}
The ``$>$'' symbol is the prompt. It's presence means that RTB is ready to
accept commands and program lines typed into it.
\index{Prompt}

\section{A quick introduction to BASIC}
(You may wish to skip this if you are familiar with BASIC)

BASIC and therefore RTB has three basic modes of operation: Immediate
mode, program store mode and program run mode.
\index{Modes}

\section{Immediate mode}
\index{Immediate mode}
This is when commands you type into the system are executed immediately.
You can use this mode to perform simple calculations, examine the state
of variables and do other ``housekeeping'' tasks such as saving and loading
your programs, and so on.

\noindent
A few examples:
\begin{verbatim}
  PRINT "Hello"
  PRINT "My name is: Gordon Henderson"
\end{verbatim}
Calculator:
\begin{verbatim}
  PRINT 1+2
  PRINT 1+2*3
\end{verbatim}
\index{Calculator}

A \meek quick note about the calculator and symbols used. There isn't a
proper divide symbol on your keyboard: $\div$ so we use the forward-slash
character instead: {\tt /} and similarly rather than use the multiply
character: $\times$ we use the asterisk: {\tt *}.

\section{Run mode}\index{Run mode}
This is simply when a program is running. To start a stored program,
use the {\tt RUN} command. The program will stop running when it
encounters a {\tt STOP} or {\tt END} instruction, or when an error is
detected. You can stop a running program at any time by using the {\tt
ESC}\footnote{Short for {\em Escape}} key at any time.

\section{Stored program mode}
\index{Stored program mode}

This mode allows us to store a program in the computers memory which we
can then execute over and over again. We differentiate immediate mode
from stored program mode by giving each line we type a line number at
the start of the line.

\noindent
To enter a simple program:
\begin{verbatim}
  10 PRINT 1 + 2 * 3
  20 END
\end{verbatim}
Remember to press the {\tt ENTER} key after every line.

To insert a line of code between two existing lines, you can use a
line-number in-between the existing numbers, so using the example above,
to insert a line we can type:
\begin{verbatim}
   5 PRINT "The answer to 1 + 2 * 3 is:"
\end{verbatim}
Verify it with the {\tt LIST} command -- and note that the computer has
added line 5 before line 10, which is what you expect because computers
are boringly good at counting. Run the program again and note the output.

\noindent
Add this line:
\begin{verbatim}
  15 PRINT "Six times seven is: "; 6 * 7
\end{verbatim}
list and run the program again. 

A few final notes here: If you enter a new line with the same number as
one already entered, it will delete the one stored and replace it with
the one you entered, or if you enter a line number and nothing else
(just press {\tt ENTER} after the line number, then it will delete any
line with that number.

It \meek is also good practice to regularly save your work.  Imagine the
frustrations of having a power cut which will lose all your work
so-far\dots

\section{BASIC}
\index{BASIC}
This is the essence of BASIC programming. You can enter commands to
be executed immediately, or enter program lines (with line numbers)
which the computer will store for as long as it's turned on, or until
you delete them by overwriting them, or using the {\tt NEW} command.

List your programs with the {\tt LIST} command and run them with the
{\tt RUN} command.

When entering new programs it is suggested to start at line number 10
and go up in 10s. That way, there is plenty of room to enter new lines
in-between if required. (But there's always the {\tt RENUMBER}
\index{RENUMBER} command if you run out of space)\\
